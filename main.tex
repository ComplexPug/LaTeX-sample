\documentclass[12pt,A4paper,oneside]{ctexbook}

\usepackage{./style/mynote}
\usepackage{./style/code}
\usepackage{./style/titlepage}

\usepackage{floatrow}

\university{NEUQ}
\coursename{Introduction~to~Artificial~Intelligence}
\title{Lecture Note}
\zhcoursename{人工智能导论}
\author{ComplexPug}
\coursedate{2023/5/2}
\instructor{Instructors: Dr.~Ins \textsc{Tructor1}\par
   	Dr.~Ins \textsc{Tructor2}, Dr.~Ins \textsc{Tructor3}\par}
\coverimage{figures/spongebob.png}

\begin{document}
	
\begin{titlepage}
   \maketitle
\end{titlepage}

\begin{titlepage}
	\tableofcontents
\end{titlepage}    

	
	
	
\section{介绍}

简单模板。
	
\section{定义}

\begin{Define}
	定义。
\end{Define}

\begin{Example}
	举例。
\end{Example}

\section{提示文本框}

\begin{mytips}[red]{结论}
    三角形的角平分线和这个角对边相交,这个顶点和交点的线段叫角平分线。
    如果一个直角三角形的斜边和一条直角边与另一个直角三角形的斜边和一条直角边对应成比例,那么这两个直角三角形相似。
\end{mytips}


\begin{mybox}{三角形相似}
如果一个三角形的两个角与另一个三角形的两个角对应相等,那么这两个三角形相似.
\begin{mybox}[red]{}
内容...
\end{mybox}
\begin{mybox}[brown]{test}
内容...
\end{mybox}
简述为:两角对应相等,两三角形相似。
\end{mybox}

\section{插入代码}

\lstinputlisting{./code/example.py}

\lstinputlisting[
	caption=1,
	label=2
]{./code/example.cpp}

\lstinputlisting[
	caption=3,
	label=4
]{./code/example.asm}

\begin{lstlisting}
#include <bits/stdc++.h>
#define FOR(i, a, b) for(int i = a; i <= b; ++i)
using namespace std;
#define int long long
const int _ = 2e6+6;

signed main() {
	ios::sync_with_stdio(false);
	cin.tie(0), cout.tie(0);
	/*注释*/
	注释
	return 0;
}
\end{lstlisting}

\section{图片}

\begin{figure}[h]
	\centering         
	\includegraphics[width=2cm]{figures/spongebob.png} 
	\includegraphics[scale=0.1]{figures/spongebob.png}
	\includegraphics[height=3cm]{figures/spongebob.png} 
	\caption*{slager|人人都是论文排版高手}
\end{figure}

\end{document}





